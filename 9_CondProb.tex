\documentclass[12pt]{article}
\usepackage{amsfonts,amssymb}
\usepackage{amsmath}
\usepackage{amsthm}
\usepackage{hyperref}
\usepackage{graphicx}
\usepackage{listings}
%\documentstyle[12pt,amsfonts]{article}
%\documentstyle{article}

\setlength{\topmargin}{-.5in}
\setlength{\oddsidemargin}{0 in}
\setlength{\evensidemargin}{0 in}
\setlength{\textwidth}{6.5truein}
\setlength{\textheight}{8.5truein}
%
%\input ../adgeomcs/lamacb.tex
%\input ../mac.tex
%\input ../mathmac.tex
%
\input xy
\xyoption{all}
\def\fseq#1#2{(#1_{#2})_{#2\geq 1}}
\def\fsseq#1#2#3{(#1_{#3(#2)})_{#2\geq 1}}
\def\qleq{\sqsubseteq}
\newtheorem{theorem}{Theorem}
%cis51109hw1

%
\begin{document}
\begin{center}
\fbox{{\Large\bf Conditional Prob}}\\
\vspace{1cm}
\end{center}

\vspace{0.5cm}\noindent


\section*{Conditional Probability}
Often times, given the occurrence of an event, the probability of another event changes. The general formula for conditional probability is

\begin{align*}
P(E|F) = \frac{P(E \cap F)}{P(F)}
\end{align*}

Example - rolling a red die and a blue die. What is the probability that the two numbers on the dice sum to at least 11 given the information that the blue die has the value 5.

$P(E \cap F) = \frac{1}{36}$ ; the only way this happens is if the red die has the value 6.

$P(F) = \frac{6}{36}$ ; when the blue die is 6, the red die could be anything.

Therefore the conditional probability is $P(E|F) = 1/6$.

\subsection*{Using conditional probability to reason about an event}
Conditional probability can be used to split up the probability of an event conditioned by the occurrence of some other event.

A student knows 80\% of the material on a true-false exam. If the student knows the material, she has a 95\% chance of getting it right. If the student does not know the material she just guesses and as expected has just a 50\% chance of getting it right.

What is the probability of getting the question right?

In this type of problem we can basically define the following events

R - the event of getting the question right

K - the event of knowing the question

$P(R) = P(R \cap K) + P(R \cap K^c)$

$P(R \cap K) = P(R|K)P(K) = 0.95 * 0.8 = 0.76$

$P(R \cap K^c) = P(R|K^c)P(K^c) = 0.5 * 0.2 = 0.1$

Adding those you get a probability of 86\% to get a question right.

\section*{Independence}

The following conditions are independent and correspond to the criteria for declaring 2 events to be independent.

\begin{align*}
P(E|F) &= P(E) \\
P(E \cap F) &= P(E)P(F) \\
P(F|E) &= P(F)
\end{align*}

\medskip

As a really small example (more in next lecture) consider the event of rolling two dice of different color (red and yellow). We want to show the event "total number of dots on top is odd" is independent of the event "the red die has an odd number of dots on top".


\subsection*{Truel}

%\begin{enumerate}
%\item Your neighbour has 2 children. You learn that he has a son, Joe. What is the probability that Joe's sibling is a
%brother?
%
%Consider the sample space in this question. You know there are 2 children so the possibilities are $\{BB,GG,BG,GB\}$. The reason we want to include both $GB$ and $BG$ is because one represents that the first born is a boy and the other represents the first boy is a girl.
%
%The event $E$ that the neighbour has a son Joe is the set $E = \{BB,BG,GB\}$.
%
%The event $F$ that the neighbour has two sons, which is the same thing as saying that Joe has a brother is just the set $F = \{BB\}$
%
%$P(F|E) = \frac{P(F \cap E)}{P(E)} = \frac{1/4}{3/4} = \frac{1}{3}$
%
%
%\item Your neighbour has 2 children. He picks one of them at random and decides to visit your house. The kid he brings is his son named Joe. What is the probability that Joe's sibling is a brother?
%
%The sample space does not change but what has changed is the way in which the gathering of information has occurred. The father has done a random sampling before you find out information.
%What changes? 
%
%Let the event $E'$ be your neighbour randomly chose a child and that happened to Joe. Event $E'$ does imply the event $E$ that we stated before. If the father randomly picks Joe then you do get the information that your neighbour has a son. However the other way round does not work. If you know that he has a son Joe, that does guarantee that he random picks Joe to bring over to your house.
%
%\begin{multline*}
%P(E') = P(E'|\{BB\}) P(\{BB\}) + P(E'|\{BG\}) P(\{BG\}) + P(E'|\{GB\})  P(\{GB\}) \\
%+ P(E'|\{GG\})P (\{GG\})
%\end{multline*}
%
%The last term is 0 so can be eliminated right away. For the others, if there is a girl and a boy there is a 0.5 chance of bringing the boy over.
%
%$P(E') = 1 * \frac{1}{4} + 0.5 * \frac{1}{4} + 0.5 * \frac{1}{4}$
%
%Therefore the conditional probability
%
%$P(F|E') = \frac{P(F \cap E')}{P(E')} = \frac{P\{BB\}}{P(E')} = \frac{1/4}{1/2} = \frac{1}{2}$


%\end{enumerate}

\section*{Bayes Theorem}

Suppose that $F$ and $X$ are events from a common sample space and $P(F) \neq 0$ and $P(X) \neq 0$. Then

\begin{align*}
P(F|X) = \frac{P(X|F)P(F)}{P(X|F)P(F) + P(X|F^c)P(F^c)}
\end{align*}

where $F^c$ is just the event of $F$ not happening.

\textbf{Example - Taken from UWash notes}

In Orange County, 51\% of the adults are males. One adult is randomly selected
for a survey involving credit card usage.
\begin{enumerate}
\item Find the prior probability that the selected person is a male.
\item It is later learned that the selected survey subject was smoking a cigar. Also, 9.5\%
of males smoke cigars, whereas 1.7\% of females smoke cigars (based on data
from the Substance Abuse and Mental Health Services Administration). Use this
additional information to find the probability that the selected subject is a male.
\end{enumerate}

Let us use some notation to simplify the analysis

$M$ = male . Thereby $M^c$ = female. 

$S$ = cigar smoker. And of course this means $S^c$ = not a cigar smoker.

$P(M) = 0.51$

The second part is basically asking us to compute $P(M|S)$.

\begin{align*}
P(M|S) = \frac{P(M) P(S|M)}{P(S|M^c) P(M^c) + P(S|M) P(M)}
\end{align*}

We are given $P(S|M)$ since 9.5\% males smoke. Similarly $P(S|M^c)$ has been provided since 1.7\% females smoke.

\begin{align*}
P(M|S) = \frac{0.51 * 0.095}{0.51 * 0.095 + 0.49 * 0.017}
= 0.853
\end{align*}

So there is an 85.3\% chance that it is a male if you observe cigars! Stands to reason. Men smoking is a far more likely event than women smoking (based on the data). 

\section*{Bayes Theorem and medical tests}
Bayes theorem is used to evaluate probabilities for medical tests. Here is a questions as an example

\medskip

There is a medical test that has been designed to screen for a disease that affects 5 in 1000 people. Suppose the false positive rate is 3\% and the false negative rate is 1\%. 

What is the probability that a randomly chosen person who tests positive actually has the disease?

\textbf{Solution}

We first need to clearly understand what false positive and false negative mean. Both these terms are used in detection systems and it is terminology that is commonly used for classifiers in machine learning as well.

False positive - the test says the person has the disease when the person actually does not have it.

False negative - the test says the person does not have the disease when the person actually does have it.

Now to solve this question, let us define some events

\begin{itemize}
\item Let $A$ be the the event that the person tests positive for the disease.
\item $B_1$ be the event that a randomly chosen person has the disease.
\item $B_2$ be the event that a randomly chosen person does not have the disease.
\end{itemize}

Now the question is asking $P(A|B_1)$.

\begin{align*}
P(B_1|A) = \frac{P(A|B_1)P(B_1)}{P(A|B_1)P(B_1) + P(A|B_2)P(B_2)}
\end{align*}

Now $P(B_1)$ is  just the probability that a randomly selected person has a disease. So 0.005.
Also $P(B_2)$ is just the probability that a randomly selected person does not have the disease. So 0.995.

$P(A|B_1)$ = 1 - probability of a false negative = 1- 0.01= 0.99

$P(A|B_2)$ = probability of a false positive = 0.03

Plugging all these values back into the original 

\begin{align*}
P(B_1|A) &= \frac{0.99*0.005}{0.099*0.005 + 0.03 * 0.995} \\
&= 0.1422 \\
&= 14.22\%
\end{align*} 

What is the probability that a randomly chosen person who tests negative does not actually have disease

This is now asking us to compute $P(B_2|A^c)$ which can again be broken by Bayes theorem as 

\begin{align*}
P(B_2|A^c) = \frac{P(A^c|B_2)P(B_2)}{P(A^c|B_2)P(B_2) + P(A^c|B_1)P(B_1)}
\end{align*}

Similar to our previous reasoning

$P(A^c|B_2)$ = 1 - probability of a false positive = 1 - 0.03 = 0.97

$P(A^c|B_1)$ = probability of a false negative = 0.01

Plugging these values into the above equation we get a probability of 99.99\%.

Therefore we have a very high probability that someone who tests negative actually does not have the disease. We have a low probability that someone who is testing positive actually has the disease. The rationale behind this type of test is that as a screening test it is more important to make sure that the negative results are truly negative. 

Those that test positive are asked to take a test where the probability of detecting the disease when the person has the disease is much higher. But those tests might be expensive and the cheaper screening test ensures that not all of the population has to pay for those expensive tests.

\section*{Application - Bayesian Machine Learning}

The primary goal of machine learning is learning models of data. 
The Bayesian framework for machine learning states that you start out by enumerating all reasonable models of the data and then assign what is called a prior probability $P(M)$ to each of these models. Then you go out and collect some data (that part has become easy these days!). 

Apply Bayes Theorem to get

\begin{align*}
P(M|D) = \frac{P(D|M)P(M)}{P(D)}
\end{align*}

Once you have observed the data $D$, you evaluate how probable the data was under each of these models to compute $P(D|M)$. Multiplying this probability by the prior and renormalizing gives you what is called the posterior probability, $P(M|D)$, which encapsulates everything that you have learned from the data regarding the models under consideration. 

How do you compare two models M and M'? We compute their relative probability given the data: $P(M)P(D|M)$ and $P(M')P(D|M')$.

This leads to a very common technique called MAP estimation which stands for maximum a posteriori. Taken from wikipedia - \href{http://en.wikipedia.org/wiki/Maximum_a_posteriori_estimation}{\underline{MAP}}.



\end{document}



