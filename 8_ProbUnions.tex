\documentclass[12pt]{article}
\usepackage{amsfonts,amssymb}
\usepackage{amsmath}
\usepackage{amsthm}
\usepackage{hyperref}
\usepackage{graphicx}
\usepackage{listings}
%\documentstyle[12pt,amsfonts]{article}
%\documentstyle{article}

\setlength{\topmargin}{-.5in}
\setlength{\oddsidemargin}{0 in}
\setlength{\evensidemargin}{0 in}
\setlength{\textwidth}{6.5truein}
\setlength{\textheight}{8.5truein}
%
%\input ../adgeomcs/lamacb.tex
%\input ../mac.tex
%\input ../mathmac.tex
%
\input xy
\xyoption{all}
\def\fseq#1#2{(#1_{#2})_{#2\geq 1}}
\def\fsseq#1#2#3{(#1_{#3(#2)})_{#2\geq 1}}
\def\qleq{\sqsubseteq}
\newtheorem{theorem}{Theorem}
%cis51109hw1

%
\begin{document}
\begin{center}
\fbox{{\Large\bf Probability of Unions}}\\
\vspace{1cm}
\end{center}

\vspace{0.5cm}\noindent

\section*{Probability of unions}
Two events are mutually exclusive if they are disjoint (which is another way of saying that their intersection is empty).

Having done set theory we know how to count the elements of a set if the two are mutually disjoint. 
Just applying that idea

$P(A \cup B) = P(A) + P(B)$.

\medskip

\textbf{Example}

Consider the situation in which files are stored on a distributed network that has 30 computers. Three copies of File 1 are stored at three distinct locations in the network, and three copies of File 2 are stored at three different locations in the network (locations for File 1 are different from locations for File 2). Suppose that there are 6 random computers that have failed. What is the probability that either file has been wiped out? Let $F_1$ be the event that all three copies of File 1 are gone and $F_2$ the event that all three copies of File 2 are gone. What is $P(F_1 \cup F_2)$?

First let us compute $P(F_1)$. Our sample space consists of all the possible ways in which 6 out of 30 computers can fail. That is the same as choosing 6 from 30 or $30 \choose 6$. For the event described, we need all 3 computers that hold $F_1$ to be wiped out but the remaining 3 failed computer could be any of the rest. So that is $27 \choose 3$ ways.

$P(F_1) = \frac{{27 \choose 3}}{{30 \choose 6}}$

$F_2$ in terms of the probability calculation is exactly the same. So $P(F_2) = P(F_1)$.

But is there a chance that both $F_1$ and $F_2$ happen? Indeed. If the 6 computers that fail are the $F_1$ and $F_2$ computers. Which can only happen in this 1 disastrous way.

Therefore,

\begin{align*}
P(F_1 \cup F_2) = \frac{2 \cdot \binom{27}{3}}{\binom{30}{6}} - \frac{1}{\binom{30}{6}}
\end{align*}


\end{document}



